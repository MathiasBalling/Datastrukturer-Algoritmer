\documentclass{article}

\usepackage[a4paper, total={6in, 9in}]{geometry}

\usepackage{amsmath} % For math
\usepackage{amssymb} % For math
\usepackage{amsthm} % For math
\usepackage{esint} % For math
\usepackage{siunitx} % For SI units

\usepackage{graphicx} % For figures
\usepackage{subcaption} % For figure

\usepackage{float} % For figure placement

\usepackage{minted} % For code

\usepackage{hyperref} % For links
\hypersetup{
	colorlinks,
	citecolor=black,
	filecolor=black,
	linkcolor=black,
	urlcolor=blue,
}
\usepackage{cleveref} % For references

\usepackage{parskip} % For paragraph spacing and no indentation

% \usepackage{tabularx} % For tables
\usepackage[table,xcdraw]{xcolor} % For tables
\def\arraystretch{1.5} % Better looking tables
\usepackage{adjustbox}

% Costum commands:

\newcommand{\qr}{{\quad\Rightarrow\quad}}
\newcommand{\qqr}{{\qquad\Rightarrow\qquad}}



\title{Porteføljeopgave 1}
\author{Mathias Balling Christiansen}
\date{\today}

\begin{document}
\maketitle
\tableofcontents
\newpage

\section{Opgave} %1
Jeg har implementeret algoritmen så ved lige tal trækkes 1 fra. 
Og ved ulige tal trækkes 2 fra for at ramme det næste ulige tal.
Derved spares uoverflødige kald. Algoritmen stopper ved n=1.

\section{Opgave} %2
Funktionen har et for-loop med 2 nestede for-loops and derefter et sequentielt
for-loop.
Det yderste for-loop fordobler i hver gang det kører, derfor kan den
estimeres med $\log{N}$.
Det midterste for-loop fordobler j hver gang det kører, derfor kan den
estimeres med $\log{N}$.
Det inderste inderste for-loop går fra 0 til $N\cdot\sqrt{N}$. 

Samlet giver dette: $N^{3/2}\cdot\log{(N)}^2$.

Efter det nestede for-loop er et for-loop der går til $N\cdot N$. Hvilket
er $N^2$.

Ved at plotte begge ses det at $N^{3/2}\cdot\log{(N)}^2$ og $N^2$
mødes ved 65536. Derfor er tidskompleksiteten:
$$
\begin{array}{cl}
  N<65536: & O(N^{3/2}\cdot\log{(N)}^2)\\
  N>65536: & O(N^2)
\end{array}
$$


\section{Opgave} %3
Jeg har implementeret algoritmen så den stopper ved at dens længde er under 3 (basecase),
da der skal bruges 3 tal til at sammenligne summen af 2 tal med den 3.
Jeg tjekker bogstavests talværdi ved at trække ascii værdien af 0 fra.
Hvis summen af 2 tal er lig med det 3. tal returneres 'true'.
Hvis længden stadig er over 3 og summen ikke gav det 3. tal,
kaldes funktionen med samme tekst, men hvor det første tal er fjernet.

\section{Opgave} %4
Her implementeres en algoritme der finder hvilke 3 tal i et array,
der giver det tætteste tal på en potens af 2.
For at optimere den har jeg et tjek der ser om vi rammer en præcis
potens af 2. Hvis vi gør det, kan vi ikke komme tættere og derfor returneres.
Tidskompleksiteten af algoritmen er $O(n^3)$, på grund af de 3 for-loops.
For at få lavere tidskompleksitet algoritmen kunne alle summene af 
to tals gemmes i et hash map. Herefter kunne der itereres over alle
værdier i dette, hvor det sidste tal summere og sammenlignes med potens af 2.
Denne algoritme ville have 2 dobbel for-loops i sekvens, hvilket er $O(n^2)$

\section{Opgave} %5
Der er 3 for-loops. Det yderste kører $\sqrt{N}$ gange.
Det midterste kører N gange.
Det inderste vil fordobler k hver gang den kører. Dette svarer til
en logoritmisk tidskompleksitet.
Til sammen bliver tidskompleksiteten $O(N\cdot\sqrt{N}\cdot \log{N})$

\section{Opgave} %6
Her implementeres en algoritme der finder summer tal der er dividerbare med 3
i intervallet [0,N].
Hvis tallet er 0 returneres 0 (basecase).
Hvis tallet er dividerbart med 3, returneres tallet og vi kalder algoritmen igen
med tallet minus 3. 
Hvis tallet ikke er dividerbart med 3, returneres tallet minus (tallet modulus 3).
Ved at trækket tallet modulus 3 fra ved vi at vi rammer et tal der er dividerbart
med 3 næste gang.

\section{Opgave} %7
Her implementeres en algoritme der finder ud af om et tal, Z, kan beskrives af $X^Y$. 
Hvis flere kombinationer er muligt skal største X værdi vælges.
Jeg kigger på alle værdier på X og Y, der er over 2 og hvor $X^Y$ er under eller lig
med resultatet. Dette gør at der spares mange udregning. Tidskompleksiteten for 
denne algoritme er $O(N^2)$

\section{Opgave} %8
Original:
\begin{table}[H]
\centering
\begin{tabular}{|l|l|}
\hline
\rowcolor[HTML]{C0C0C0} 
Index & Værdi \\ \hline
0     &       \\ \hline
1     &       \\ \hline
2     & V     \\ \hline
3     & R     \\ \hline
4     &       \\ \hline
5     &       \\ \hline
6     & P     \\ \hline
7     &       \\ \hline
8     & E     \\ \hline
9     &       \\ \hline
10    & F     \\ \hline
\end{tabular}
\end{table}

Efter Q til indeks 7:
\begin{table}[H]
\centering
\begin{tabular}{|l|l|}
\hline
\rowcolor[HTML]{C0C0C0} 
Index & Værdi \\ \hline
0     &       \\ \hline
1     &       \\ \hline
2     & V     \\ \hline
3     & R     \\ \hline
4     &       \\ \hline
5     &       \\ \hline
6     & P     \\ \hline
7     & Q     \\ \hline
8     & E     \\ \hline
9     &       \\ \hline
10    & F     \\ \hline
\end{tabular}
\end{table}

Efter C til indeks 8. Indeks 8 er optaget så vi lægger $1^2$ til:
\begin{table}[H]
\centering
\begin{tabular}{|l|l|}
\hline
\rowcolor[HTML]{C0C0C0} 
Index & Værdi \\ \hline
0     &       \\ \hline
1     &       \\ \hline
2     & V     \\ \hline
3     & R     \\ \hline
4     &       \\ \hline
5     &       \\ \hline
6     & P     \\ \hline
7     & Q     \\ \hline
8     & E     \\ \hline
9     & C     \\ \hline
10    & F     \\ \hline
\end{tabular}
\end{table}
Efter H til indeks 2. Indeks 2 er optaget så vi lægger $1^2$ til, men 3 er også
optaget. Så vi lægger $2^2$ til, men 6 er også optaget. Så vi lægger $3^2$ til, 
her ender vi på 0:
\begin{table}[H]
\centering
\begin{tabular}{|l|l|}
\hline
\rowcolor[HTML]{C0C0C0} 
Index & Værdi \\ \hline
0     & H     \\ \hline
1     &       \\ \hline
2     & V     \\ \hline
3     & R     \\ \hline
4     &       \\ \hline
5     &       \\ \hline
6     & P     \\ \hline
7     & Q     \\ \hline
8     & E     \\ \hline
9     & C     \\ \hline
10    & F     \\ \hline
\end{tabular}
\end{table}

\section{Opgave} %9
Denne funktion tager et tal n og kalder funktionen rekursivt med n-1 og n-2.
Det vil sige funktionen har tidskompleksiteten $O(2^N)$, fordi den kalder sig 
selv 2 gang pr. rekursion. Fx n=10 til rekursion kalder n=9 og n=8. 
n=9 kalder derefter n=8 og n=7. n=8 kalder n=7 og n=6. Så $1 \to 2 \to 4 \to 8$
Dvs der ender med at være $O(2^N)$ tidskompleksitet.

\section{Opgave} %10
I denne opgave er mit basecase at $N<2$. Her returneres 0. For alle tal over returneres
1 plus et rekursivt kald med $N/2$. Derved går vi fra functionskaldene 
$32 \to 16 \to 8 \to 4 \to 2$ hvilket giver $1+1+1+1+1+0=5$.

\section{Opgave} %11
Jeg starter med at lave et tomt hash map. Herefter går jeg igennem 
stemmerne. En stemme fx 7 er en key og dens værdi er antal gange den key
er forkommet. Hvis det første gang vi ser en key sættes værdien til 1 og 
eller bliver den incremented. Undervejs tjekkes om den indsatte kandidat har over
halvdelen af stemmerne og hvis den har det returneres true, da kun en kan have det.

Min tidskompleksitet for algoritmen er O(N) da vi kun går igennem algoritmen n
gang.

\end{document}
