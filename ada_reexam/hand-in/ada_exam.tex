\documentclass{article}

\input{preamble.tex}


\title{ADA reexam}
\author{Mathias Balling Christiansen}
\date{\today}

\begin{document}
\maketitle
\tableofcontents
\newpage

\section{Opgave} %1
\begin{minted}{cpp}
bool sumAfToTalLigParameterV1(int arr[], int len, int X) {
  // O(n^2)
  for (int i = 0; i < len - 1; i++) {
    for (int j = 1; j < len; j++) {
      int sum = arr[i] + arr[j];
      if (sum == X) {
        return true;
      }
    }
  }
  return false;
}
bool sumAfToTalLigParameterV2(int arr[], int len, int X) {
  // O(n)
  int right_idx = len - 1;
  int left_idx = 0;
  while (true) {
    int sum = arr[left_idx] + arr[right_idx];
    if (sum == X) {
      return true;
    } else if (sum > X) {
      right_idx--;
    } else if (sum < X) {
      left_idx++;
    }
    if (left_idx == right_idx) {
      return false;
    }
  }
  return false;
}
\end{minted}
\section{Opgave} %2
1. Første loop
\begin{minted}{cpp}
for (int i = 0; i < 10000; i++) { 
    for (int j = 0; j < N*10; j++) { 
        for (int k = N; k > 0; k = k/10) { 
            x++; 
        } 
    } 
}
\end{minted}

Ydre loop:

- Løber over \( i \) fra \( 0 \) til \( 10000 \), hvilket giver \( O(1) \).

Mellemste loop:

- \( j \) løber fra \( 0 \) til \( 10*N \), hvilket giver \( O(N) \)

Indre loop:

- \( k \) starter ved \( N \) og divideres med 10 i hver iteration (\( k = k / 10 \)).\\
- Antallet af iterationer for dette loop er \( \log_10(N) \), hvilket giver \( O(\log N) \).

Samlet tid for de tre loops:

- Den totale tid for denne del af koden er produktet af iterationerne:
  \[
  O(1) \cdot O(N) \cdot O(\log(N)) = O(N \cdot \log(N))
  \]

\section{Opgave} %3
\begin{minted}{cpp}
const std::string vowels = "aeiouy";
int antalVokaler(std::string str, int l) {
  // Basecase
  if (l < 0) {
    return 0;
  }
  if (vowels.find(str[0]) != std::string::npos) {
    return 1 + antalVokaler(str.substr(1), l - 1);
  } else {
    return antalVokaler(str.substr(1), l - 1);
  }
}
\end{minted}
\section{Opgave} %4
\begin{minted}{cpp}
void minSortering(int arr[100]) {
  int count_arr[201] = {};
  for (size_t i = 0; i < 100; i++) {
    count_arr[arr[i]]++;
  }
  // 0 -> 200 is always constant O(1)
  for (size_t i = 0; i < 201; i++) {
    // O(N)
    for (size_t j = 0; j < count_arr[i]; j++) {
      std::print("{} ", i);
    }
  }
  std::println("");
}
\end{minted}
\section{Opgave} %5
\{0,9,23,106,10,36,38,98,84,12,14,50,55,35,68\}

%           9
%        /       \
%      23         106
%    /   \      /    \
%  10    36    38    98
% / \   / \   / \    / 
%84 12 14 50 55 35 68
Opfyld max-heap fra bunden. 
\begin{itemize}
  \item Byt 10 og 84
  \item Byt 50 og 36
  \item Byt 38 og 55
%           9
%        /       \
%      23         106
%    /   \      /    \
%  84    50    55    98
% / \   / \   / \    / 
%10 12 14 36 38 35 68
  \item Byt 84 og 23
  \item Byt 106 og 9
  \item Byt 98 og 9
  \item Byt 68 og 9
\end{itemize}
%           106
%        /       \
%      84         98
%    /   \      /    \
%  23    50    55    68
% / \   / \   / \    / 
%10 12 14 36 38 35  9

Endeligt array:
\{0,106,84,98,23,50,55,68,10,12,14,36,38,35,9\}

Træet ville være komplet da det er fyldt op fra venstre til højre,
men ikke udfyldt helt på sidste niveau for at være perfekt.

\section{Opgave} %6
DEMOCRAT:
\begin{itemize}
  \item D=\( 11*4\%16=12 \)
  \item E=\( 11*5\%16=7 \)
  \item M=\( 11*13\%16=15 \)
  \item O=\( 11*15\%16=5 \)
  \item C=\( 11*3\%16=1 \)
  \item R=\( 11*18\%16=6 \)
  \item A=\( 11*1\%16=11 \)
  \item T=\( 11*20\%16=12 \)
\end{itemize}

\begin{table}[H]
\centering
\begin{tabular}{|l|l|}
\hline
\rowcolor[HTML]{C0C0C0} 
Index    & Value \\ \hline
0        &      \\ \hline
1        & C     \\ \hline
2        &      \\ \hline
3        &      \\ \hline
4        &      \\ \hline
5        & O     \\ \hline
6        & R     \\ \hline
7        & E     \\ \hline
8        &      \\ \hline
9        &      \\ \hline
10       &      \\ \hline
11       & A     \\ \hline
12       & D     \\ \hline
13       & T     \\ \hline
14       &      \\ \hline
15       & M     \\ \hline
\end{tabular}
\end{table}
REPUBLICAN:
\begin{itemize}
  \item R=\( 11*18\%16=6 \)
  \item E=\( 11*5\%16=7 \)
  \item P=\( 11*16\%16=0 \)
  \item U=\( 11*21\%16=7 \)
  \item B=\( 11*2\%16=6 \)
  \item L=\( 11*12\%16=4 \)
  \item I=\( 11*9\%16=3 \)
  \item C=\( 11*3\%16=1 \)
  \item A=\( 11*1\%16=11 \)
  \item N=\( 11*14\%16=10 \)
\end{itemize}

\begin{table}[H]
\centering
\begin{tabular}{|l|l|}
\hline
\rowcolor[HTML]{C0C0C0} 
Index    & Value \\ \hline
0        & P    \\ \hline
1        & C     \\ \hline
2        &      \\ \hline
3        & I    \\ \hline
4        & L    \\ \hline
5        &       \\ \hline
6        & R      \\ \hline
7        & E      \\ \hline
8        & U    \\ \hline
9        &     \\ \hline
10       & B    \\ \hline
11       & A     \\ \hline
12       &       \\ \hline
13       &       \\ \hline
14       & N    \\ \hline
15       &       \\ \hline
\end{tabular}
\end{table}

\section{Opgave} %7
Using Dijkstras Algorithm:
\begin{table}[H]
\centering
\begin{tabular}{|l|l|l|l|}
\hline
\rowcolor[HTML]{C0C0C0} 
v  & Known & $d_v$   & $p_v$     \\ \hline
S  & true  & 0    & 0    \\ \hline
A  & true  & 4    & H   \\ \hline
B  & true  & 9    & F   \\ \hline
C  & true  & 16   & G   \\ \hline
D  & true  & 20   & C   \\ \hline
E  & true  & 11   & H   \\ \hline
F  & true  & 7    & A   \\ \hline
G  & true  & 13   & E   \\ \hline
H  & true  & 2    & S    \\ \hline
\end{tabular}
\end{table}

Jeg bruger Kruskal's. Rækkefølge følger tabellen.
\begin{table}[H]
\centering
\begin{tabular}{|l|l|}
\hline
\rowcolor[HTML]{C0C0C0} 
Weight  & Node-pair \\ \hline
2       & (S,H)     \\ \hline
2       & (H,A)     \\ \hline
2       & (H,F)     \\ \hline
2       & (F,B)     \\ \hline
2       & (E,G)     \\ \hline
3       & (G,C)     \\ \hline
4       & (C,D)     \\ \hline
5       & (B,E)     \\ \hline
\end{tabular}
\end{table}
Total vægt: 22

\end{document}
