\documentclass{article}

\input{preamble.tex}


\title{ADA exam}
\author{190483865}
\date{\today}

\begin{document}
\maketitle
\tableofcontents
\newpage

\section{Opgave 1}
\begin{minted}{cpp}
// Jeg går ud fra at index start ved længden af ordet minus 1.
// Eksempel kald:
// std::string word = "banana";
// countLettersInWord(word, 'a', word.length() - 1);

int countLettersInWord(std::string word, char letter, int index) {
  if (index < 0) {
    return 0;
  }
  if (word[index] == letter) {
    return 1 + countLettersInWord(word, letter, index - 1);
  } else {
    return countLettersInWord(word, letter, index - 1);
  }
}
\end{minted}

\section{Opgave 2}
For loop 1:

Yderste for loop har: $O(\log{N})$ pga. $i+=i$.

Mellemste har $O({\log^2{N}})$ da ${\log^2{N}}\sim 2\cdot{\ln^2{N}}$

Inderste har: $O(\sqrt{N})$, da $k$ kører til kvadratroden af $N$.

For loop 2:

Tidskompleksitet $O(N\cdot \sqrt{N})$

\line(1,0){\textwidth}

Værdier fra test:\\
N:300,x:5346,y:5197\\
N:301,x:5346,y:5223\\
N:302,x:5346,y:5249\\
N:303,x:5346,y:5275\\
N:304,x:5346,y:5301\\
N:305,x:5346,y:5327\\
N:306,x:5346,y:5353\\
N:307,x:5346,y:5380\\
N:308,x:5346,y:5406\\
N:309,x:5346,y:5432\\
N:310,x:5346,y:5459

Det ses at ved værdier over $N>=306$ er 2. for loop langsommere end det første. 
Dvs. $N<306$ er 2. for loop hurtigere end det første. 

\line(1,0){\textwidth}

Den samlede tidskompleksitet er 
$$
\begin{array}{cl}
  N>=306: & O(N\cdot \sqrt{N})\\ 
  N<306: & O(\log^3 {N}\cdot  \sqrt{N})
\end{array}
$$
\section{Opgave 3}
Nyt kode er har "// Ny" efter sig.
\begin{minted}{cpp}
int main(){
  // Samme kode oppe over
  string minTask = "";// Ny
  string finalMaxSlack = "";// Ny
  int minVarighedAktuelEvent = numeric_limits<int>::max();// Ny (#include <limits>)
  int maxSlack = 0;// Ny

  while (true) {
    while (indeks < tabel.size() && tabel[indeks].getEvent() == aktuelEvent) {
      if (maxVarighedAktuelEvent < tabel[indeks].getDuration()) {
        maxVarighedAktuelEvent = tabel[indeks].getDuration();
        maxTask = tabel[indeks].getTask();
      }
      if (minVarighedAktuelEvent > tabel[indeks].getDuration()) { // Ny
        minVarighedAktuelEvent = tabel[indeks].getDuration();// Ny
        minTask = tabel[indeks].getTask();// Ny
      }// Ny

      indeks++;
    }
    int slackAktuelEvent = maxVarighedAktuelEvent - minVarighedAktuelEvent;// Ny
    if (slackAktuelEvent > maxSlack) {// Ny
      maxSlack = slackAktuelEvent;// Ny
      finalMaxSlack = minTask;// Ny
    }// Ny

    laengdeKritiskVej += maxVarighedAktuelEvent;
    kritiskVej += maxTask;
    maxVarighedAktuelEvent = 0;
    minVarighedAktuelEvent = numeric_limits<int>::max();// Ny
    maxTask = "";
    minTask = ""; // Ny

    if (indeks == tabel.size())
      break;
    aktuelEvent = tabel[indeks].getEvent();
  }
  cout << "Max Slack: " << finalMaxSlack << " : " << maxSlack << endl;

  // Samme kode efter dette
}
\end{minted}

\section{Opgave 4}
\begin{minted}{cpp}
void BinarySearchTree::printRoute(int value) {
  std::string route = findRoute(root, value);
  if (route != "") {
    cout << route << endl;
  } else {
    cout << "No route to " << value << " found." << endl;
  }
}

string BinarySearchTree::findRoute(BinaryNode *root, int value) {
  if (!root)
    return "";
  if (root->element == value) {
    return to_string(value);
  }
  std::string left = findRoute(root->left, value);
  std::string right = findRoute(root->right, value);
  if (left.find(to_string(value)) != std::string::npos) {
    return to_string(root->element) + " " + left;
  } else if (right.find(to_string(value)) != std::string::npos) {
    return to_string(root->element) + " " + right;
  }
  return findRoute(root->left, value) + findRoute(root->right, value);
}
\end{minted}

\section{Opgave 5}
Brug af Dijkstras Algorithm:
\begin{table}[H]
\centering
\begin{tabular}{|l|l|l|l|}
\hline
\rowcolor[HTML]{C0C0C0} 
v  & Known & $d_v$   & $p_v$     \\ \hline
F  & true  & 0    & 0    \\ \hline
A  & true  & 15   & B    \\ \hline
B  & true  & 14   & D    \\ \hline
C  & true  & 16   & A    \\ \hline
D  & true  & 5    & E    \\ \hline
E  & true  & 4    & F    \\ \hline
G  & true  & 34   & J    \\ \hline
H  & true  & 21   & C    \\ \hline
I  & true  & 19   & A    \\ \hline
J  & true  & 24   & I    \\ \hline
\end{tabular}
\end{table}

Jeg bruger Kruskal's. Rækkefølge følger tabellen.
\begin{table}[H]
\centering
\begin{tabular}{|l|l|}
\hline
\rowcolor[HTML]{C0C0C0} 
Vægt  & Node-par \\ \hline
1       & (E,D)     \\ \hline
1       & (B,A)     \\ \hline
1       & (A,C)     \\ \hline
2       & (D,I)     \\ \hline
4       & (F,E)     \\ \hline
4       & (A,I)     \\ \hline
5       & (C,H)     \\ \hline
5       & (I,J)     \\ \hline
10      & (J,G)     \\ \hline
\end{tabular}
\end{table}

Samlet vægt er: 33

\section{Opgave 6}

Y0=3\\
Y1=4\\
Y2=7\\
Y3=12\\
Y4=6\\
Y5=2\\
Y6=0

De samme tal fortsætter herefter...

Y7=0\\
Y8=2\\
Y9=6\\
Y10=12

Sådan vil tabellen se ud:
\begin{table}[H]
\centering
\begin{tabular}{|l|l|}
\hline
\rowcolor[HTML]{C0C0C0} 
Index    & Item \\ \hline
0        & Y6    \\ \hline
1        &      \\ \hline
2        & Y5    \\ \hline
3        & X    \\ \hline
4        & Y1   \\ \hline
5        &      \\ \hline
6        & Y4     \\ \hline
7        & Y2     \\ \hline
8        &     \\ \hline
9        &    \\ \hline
10       &     \\ \hline
11       &      \\ \hline
12       & Y3     \\ \hline
\end{tabular}
\end{table}


\section{Opgave 7}
Dette skulle representere en min-heap det vil sige alle tal under skal være større. 2 bryder dette:

27 på index 11 er mindre end den prioritet oppeover på 28.

18 på index 15 er mindre end den prioritet oppeover på 26.

Derfor er dette ikke en prioritetskø.

\end{document}
