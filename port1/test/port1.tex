\documentclass{article}

\usepackage[a4paper, total={6in, 9in}]{geometry}

\usepackage{amsmath} % For math
\usepackage{amssymb} % For math
\usepackage{amsthm} % For math
\usepackage{esint} % For math
\usepackage{siunitx} % For SI units

\usepackage{graphicx} % For figures
\usepackage{subcaption} % For figure

\usepackage{float} % For figure placement

\usepackage{minted} % For code

\usepackage{hyperref} % For links
\hypersetup{
	colorlinks,
	citecolor=black,
	filecolor=black,
	linkcolor=black,
	urlcolor=blue,
}
\usepackage{cleveref} % For references

\usepackage{parskip} % For paragraph spacing and no indentation

% \usepackage{tabularx} % For tables
\usepackage[table,xcdraw]{xcolor} % For tables
\def\arraystretch{1.5} % Better looking tables
\usepackage{adjustbox}

% Costum commands:

\newcommand{\qr}{{\quad\Rightarrow\quad}}
\newcommand{\qqr}{{\qquad\Rightarrow\qquad}}



\title{Porteføljeopgave 1}
\author{Mathias Balling Christiansen}
\date{\today}

\begin{document}
\maketitle
\tableofcontents
\newpage

\section{Opgave} %1
Jeg har implementeret algoritmen så ved lige tal trækkes 1 fra. 
Og ved ulige tal trækkes 2 fra for at ramme det næste ulige tal.
Derved spares uoverflødige kald. Algoritmen stopper ved n=1.

\section{Opgave} %2

\section{Opgave} %3
Jeg har implementeret algoritmen så den stopper ved at dens længde er under 3 (basecase),
da der skal bruges 3 tal til at sammenligne summen af 2 tal med den 3.
Jeg tjekker bogstavests talværdi ved at trække ascii værdien af 0 fra.
Hvis summen af 2 tal er lig med det 3. tal returneres 'true'.
Hvis længden stadig er over 3 og summen ikke gav det 3. tal,
kaldes funktionen med samme tekst, men hvor det første tal er fjernet.

\section{Opgave} %4
Her implementeres en algoritme der finder hvilke 3 tal i et array,
der giver det tætteste tal på en potens af 2.
For at optimere den har jeg et tjek der ser om vi rammer en præcis
potens af 2. Hvis vi gør det, kan vi ikke komme tættere og derfor returneres.
Tidskompleksiteten af algoritmen er $O(n^3)$, på grund af de 3 for-loops.
For at få lavere tidskompleksitet algoritmen kunne alle summene af 
to tals gemmes i et hash map. Herefter kunne der itereres over alle
værdier i dette, hvor det sidste tal summere og sammenlignes med potens af 2.
Denne algoritme ville have 2 dobbel for-loops i sekvens, hvilket er $O(n^2)$

\section{Opgave} %5

\section{Opgave} %6
Her implementeres en algoritme der finder summer tal der er dividerbare med 3
i intervallet [0,N].
Hvis tallet er 0 returneres 0 (basecase).
Hvis tallet er dividerbart med 3, returneres tallet og vi kalder algoritmen igen
med tallet minus 3. 
Hvis tallet ikke er dividerbart med 3, returneres tallet minus (tallet modulus 3).
Ved at trækket tallet modulus 3 fra ved vi at vi rammer et tal der er dividerbart
med 3 næste gang.
% Code snippet
\begin{minted}{cpp}
int logTo(int N) {
  if (N < 2)
    return 0;
  return 1 + logTo(N / 2);
}
\end{minted}


\section{Opgave} %7
% Whole file
\inputminted{cpp}{../port_10.cpp}

\section{Opgave} %8

\section{Opgave} %9

\section{Opgave} %10

\section{Opgave} %11

\section{Opgave} %12


\end{document}
