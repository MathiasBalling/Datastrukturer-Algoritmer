\documentclass{article}

\input{preamble.tex}


\title{Porteføljeopgave 1}
\author{Mathias Balling Christiansen}
\date{\today}

\begin{document}
\maketitle
\tableofcontents
\newpage

\section{Opgave} %1
Koden:
\begin{minted}{cpp}
std::string highest_word_count(std::string input) {
  if (input.empty()) {
    return "";
  }
  std::unordered_map<std::string, int> word_count;
  while (!input.empty()) {
    size_t dilimiter_index = input.find(' ');
    std::string word = "";
    if (dilimiter_index != std::string::npos) {
      word = input.substr(0, dilimiter_index);
      // Make word lowercase
      for (auto &c : word) {
        c = std::tolower(c);
      }
      input = input.substr(dilimiter_index + 1);
    } else {
      // Last word here
      word = input;
      // Make word lowercase
      for (auto &c : word) {
        c = std::tolower(c);
      }
      input.erase();
    }
    if (word.find(',') != std::string::npos ||
        word.find('.') != std::string::npos) {
      // Dilimiter will always be in the end.
      word.erase(word.end() - 1);
    }
    if (word_count.contains(word)) {
      word_count[word]++;
    } else {
      word_count[word] = 1;
    }
  }

  std::pair<std::string, int> highest_word_count = {"", 0};
  for (const auto &word_pair : word_count) {
    if (word_pair.second > highest_word_count.second) {
      highest_word_count.first = word_pair.first;
      highest_word_count.second = word_pair.second;
    }
  }

  return highest_word_count.first;
}
\end{minted}

Koden kører igennem input string en gang hvilket har tidskompleksitet $O(n)$.

Jeg bruger derefter \textbf{find} til at finde næste mellemrum hvilke også har 
tidskompleksitet $O(n)$.

Her ligger jeg de ord jeg finder ind i et hashmap og til allersidst finder jeg ordet med højeste værdi.

\section{Opgave} %2
\begin{minted}{cpp}
BinaryNode *BinarySearchTree::getOnlyChild(BinaryNode *t) const {
  if (t->left != nullptr && t->right == nullptr)
    return t->left;
  else if (t->right != nullptr && t->left == nullptr) {
    return t->right;
  } else {
    return nullptr;
  }
}

int BinarySearchTree::countBranches(BinaryNode *t) const {
  if (t == nullptr)
    return 0;
  if (BinaryNode *X = getOnlyChild(t)) { // No siblings (X)
    if (BinaryNode *X_Child = getOnlyChild(X)) { // Only child (X's child)
      if (BinaryNode *X_Child_Child = getOnlyChild(X_Child)) { // Only child (X's child's child)
        if (X_Child_Child->left == nullptr && 
           X_Child_Child->right == nullptr) { // Leaf (X's child's child and leaf)
          return 1;
        }
      }
    }
  }
  return countBranches(t->left) + countBranches(t->right);
}
\end{minted}

\textbf{Supplerende opgave 1:}

Hvad er træet:\\
Det er et binært søgetræ. Det er ikke komplet eller balanceret da 
alle noder ikke er fyldt fra venstre til højre og forskellen i dybden er 
mere end 1.

Sædvanlige oplysninger:\\
- Højde: 6 (7 $\to$ 28 $\to$ 55 $\to$ 51 $\to$ 48 $\to$ 40 $\to$ 35)\\
- Internal path length: $0 + 2 * 1 + 2 * 2 + 3 * 3 + 4 * 4 + 5 * 2 + 6 = 47$\\
- Antal noder: 15\\
- Antal leafs: 4\\
- Antal fulle noder: 3

Optimale højde + matematisk udtryk:\\
Den optimale højde er $h = \log{(n + 1)}$\\
Da vi har $n=15$ elementer så er optimale højde $h = \log{(16)}=4$

\textbf{Supplerende opgave 2:}

Fra binære søgetræ til prioritetskø (trin + tidskompleksitet):\\
Jeg ville bruge en level order traversering, hvor jeg putter alle elementer i en
prioritetskø i den rækkefølge level order vil give os. Tidskompleksiteten af
denne opgave er $O(n)$.
\section{Opgave} %3
\textbf{List rækkefølgen I hvilken noderne vil blive besøgt I en en in order og i en level order traversering.}

In order: 1 2 3 9 11 13 17 25 57 90

Level order: 11 2 13 1 9 57 3 25 90 17

\textbf{Hvad er træets internal path length?}  
$$0 + 2 * 1 + 3 * 2 + 3 * 3 + 4 = 21$$

\textbf{balancen i træet er ved node 13, hvor højden af det højre subtræ er 3, og det venstre subtræ er 0.
Hvordan kan man omarrangere noderne i det højre subtræ, så hele træet bliver et AVL-træ?}

Man kan rykke indsætte 17 på 13's plads og derefter rykke 13 til at være i venstre subtræ af 17.
 
\textbf{Kunne træet have været et AVL-træ før den seneste operation (insert eller delete, men ikke rotation)? 
Eksempler på seneste operation kunne være indsættelse af node 3 eller sletning af node 12 (venstre barn af 
node 13). Begrund dit svar.}

Nej, Der er ikke noget man kunne have indsat eller fjernet for at det var et AVL træ. Lige meget hvad ville mere end 1.

\section{Opgave} %4
\textbf{List rækkefølgen i hvilken noderne besøges i en post order og i en pre order traversering.}

Post order: 1 8 5 15 12 10 22 20 28 30 38 45 50 48 40 36 25

Pre order: 25 20 10 5 1 8 12 15 22 36 30 28 40 38 48 45 50

\textbf{Hvad er træets internal path length?}
$$0 + 2 * 1 + 4 * 2 + 5 * 3 + 5 * 4 = 45$$
\textbf{Er det et AVL-træ? Hvorfor eller hvorfor ikke?}

Nej, der er ubalance i 20, da det venstre sub-træ går 3 ned og det højre kun går 1 ned.

\section{Opgave} %5

Jeg bruger Kruskal's 
\begin{table}[H]
\centering
\begin{tabular}{|l|l|}
\hline
\rowcolor[HTML]{C0C0C0} 
Weight  & Node-pair \\ \hline
1       & (0,1)     \\ \hline
1       & (0,4)     \\ \hline
1       & (3,6)     \\ \hline
1       & (3,7)     \\ \hline
2       & (0,5)     \\ \hline
2       & (9,10)     \\ \hline
3       & (2,5)     \\ \hline
3       & (5,8)     \\ \hline
3       & (7,11)     \\ \hline
4       & (8,9)     \\ \hline
5       & (10,11)     \\ \hline
\end{tabular}
\end{table}

Totale vægt: 26

\section{Opgave} %6
\begin{table}[H]
\centering
\begin{tabular}{|l|l|l|l|}
\hline
\rowcolor[HTML]{C0C0C0} 
v  & Known & $d_v$  & $p_v$     \\ \hline
A  & true  & 0      & 0    \\ \hline
B  & true  & 5      & A    \\ \hline
C  & true  & 3      & A    \\ \hline
D  & true  & 9      & E    \\ \hline
E  & true  & 7      & G    \\ \hline
F  & true  & 8      & E    \\ \hline
G  & true  & 6      & B    \\ \hline
\end{tabular}
\end{table}



\end{document}
